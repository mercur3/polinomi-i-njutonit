\documentclass[12pt, margin=1in, a4paper]{article}

\usepackage[utf8]{inputenc}
\usepackage{indentfirst}
\usepackage[ddmmyyyy]{datetime}
\usepackage{hyperref}
\usepackage{graphicx}
\usepackage{listings}
\usepackage{xcolor}
\usepackage{amsmath}


\renewcommand{\dateseparator}{.}
\renewcommand{\contentsname}{Permbajtja}
\lstloadlanguages{Octave}
\lstset{
  basicstyle=\ttfamily\small,
  language=Octave,
  numbers=left,
  breaklines=true,
  showstringspaces=false,
  columns=flexible,
  keepspaces=true
}

\title{Polinomi interpolues i Njutonit me diferenca t:e prapme}
\author{
    Andri Reveli - \href{mailto:andri.reveli@fshnstudent.info}{\texttt{andri.reveli@fshnstudent.info}}
    \and
    Anxhela Kamberaj - \href{mailto:anxhela.kamberaj@fshnstudent.info}{\texttt{anxhela.kamberaj@fshnstudent.info}}
    \and
    Naile Mucaj - \href{mailto:naile.mucaj@fshnstudent.info}{\texttt{naile.mucaj@fshnstudent.info}}
    \and
}

\begin{document}
\maketitle
\tableofcontents

\newpage

\section{Hyrje}
  \subsection{Polinomi interpolues i Njutonit}
  Polinomi interpolues i Njutonit :esht:e nj:e metod:e p:er interpolimin e
  funksioneve t:e dh:en:e midis 2 pikave t:e caktuara. Ne form:en e p:ergjith:esuar ai
  shprehet si:
  \begin{equation}\label{Newton}
    P(x) = f(x_0) + (x - x_0)f[x_0, \ x_1] + \ \dots \ + (x - x_0)(x - x_1) \dots (x - x_{n - 1})f[x_0, \ x_1, \ \dots, \ x_n]
  \end{equation}

  P:er ta b:er:e formul:en me te qart:e mund ta shkruajm:e si:
  \[
    P(x) = f(x_0) + \sum_{i = 1}^{n}{\prod_{j = 0}^{i - 1}{(x - x_j)}f[x_0, \ \dots, \ x_i]} \\
  \]

  Dhe gabimi jepet nga formula:
  \[
    E(x) = f[x_0, \ x_1, \ \dots, \ x_n]\prod_{i = 0}^n{(x - x_i)} \ \land \ f[x_0, \ x_1, \ \dots, \ x_n] = \frac{f^n(\zeta)}{n!}
  \]

  \subsection{Diferencat e ndara}
  Supozojm:e se largesa midis pikave \(x_0, \ x_1, \ \dots, \ x_n\) :esht:e e
  barabart:e. P:er rrjedhoj:e, mund te shkruajm:e:
  \[
    h = x_i - x_{i - i}, \ i = 1, \ 2, \ \dots, \ n
  \]

  Ne k:eto kushte diferenc:e e fundme e prapme e rendit t:e par:e quajm:e:
  \(\nabla y_i = y_i - y_{i - 1}\), kurse diferenc:e t:e fundme t:e rendit \(n\):
  \(\nabla^ny_i = \nabla(\nabla^{n - 1}y_i)\).

  \subsection{Polinomi interpolues me diferenca t:e ndara}
  Duke z:evendesuar nga polinomi i Njutonit (\ref{Newton}) diferenca e ndara me
  ato t:e fundme t:e prapme marrim:
  \begin{equation}
    P_n(x) = f(x_n) + \nabla f(x_n)\frac{x - x_n}{h} + \nabla^2f(x_n)\frac{(x - x_n)(x - x_{n - 1})}{2 \cdot h!} + \dots + \nabla^n f(x_n)\frac{(x - x_n)(x - x_{n - 1}) \dots (x - x_1)}{n! \cdot h^n}
  \end{equation}


\section{Zbatime}

\section{Algoritmi ne Octave}

\section{Literatura}
\begin{enumarate}
  \item Hoxha.F, Metoda të Analizës Numerike, Infobotues, Tiranë, 2008
\end{enumarate}

